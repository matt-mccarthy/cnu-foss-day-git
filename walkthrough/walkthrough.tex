\documentclass[notitlepage]{simple}

\author{Matt McCarthy}
\title{How To Git}
\date{12 March, 2016}

\begin{document}

\maketitle

\section{Version Control}

% TODO: Get the information to put here.

\section{Git 101}

To start, we need to configure \verb|git|.
To do so, we run the following two commands.
\begin{center}
	\begin{tabular}{l}
		\verb|git config --global user.name "Your Name Here"|\\
		\verb|git config --global user.email "your.email@host.domain"|
	\end{tabular}
\end{center}
These commands tell our \verb|git| installation who we are and how to contact us so that other users know who is responsible for each commit.

Our first task is to make a repository on our local machine.
Next we want a directory in which we will store our \verb|git| repositories.
For the sake of simplicity, let's just make a new folder in the home directory called \verb|git| (e.g. run \verb|mkdir ~/git|).
We now need to \verb|cd| into our new directory, so we run \verb|cd ~/git|.
Now run \verb|mkdir my-git-repo| and then \verb|cd my-git-repo|.
We will now turn this folder into a \verb|git| repository by running \verb|git init|.
And now we have a \verb|git| repository.

We're now going to start making changes, tracking them, and committing them.
Let's begin by creating a file and telling \verb|git| to track it.
Run \verb|touch file.txt|, this will create a file called \verb|file.txt|.
If we run \verb|git status|, it will list \verb|file.txt| as an untracked file.
We now need to run \verb|git add .|.
The previous command tracks all untracked files and tracks any changes you made.
If we run \verb|git status| again, we will see that \verb|new file: file.txt| is in the list of changes to be committed.
Lastly, to commit our changes, we run \verb|git commit -m "Added file.txt"|.
This logs our changes and gives us a point to which we can revert.
If we run \verb|git status| once more, it will report that there is nothing to commit and that the working directory is clean.
The \verb|git add .| and \verb|git commit -m "message here"| commands define the workflow on a single machine, that is these commands track and log each change you make to your project.

Next we'll make some changes to the project and then reset them.
For now, run the following command.
\begin{center}
	\verb|echo "Subversion is the best version control software" > file.txt|
\end{center}
Then track the change using \verb|git add .| but don't commit the change just yet.
Run \verb|git status| and ensure that the change is tracked.
You see, the statement we piped into \verb|file.txt| is a lie and so we need to undo the change even though we just tracked it.
To do so, we run \verb|git reset --hard|.
Now run \verb|git status| to make sure the change is gone.
More generally, we can run \verb|git reset --hard commit-id| to reset to any given commit.
Since the old statement is gone, we will go ahead and pipe the correct statement into \verb|file.txt|.
\begin{center}
	\verb|echo "git is the best version control software" > file.txt|
\end{center}
We will now track and commit our changes in one command by running \verb|git commit -am "Did a thing"|.
This command adds all changes in \textit{tracked} files and then commits them in one fell swoop.
Furthermore, it does not add any untracked files, so any new files would be skipped by this command.

\section{Git Demonstration}

\section{Advanced Git}

\end{document}
